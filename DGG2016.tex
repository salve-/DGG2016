\documentclass[compress=false,usepdftitle=false, subsection=false,xcolor=dvipsnames]{beamer}
%
%
\usetheme{LMU}
\setbeamercovered{dynamic} % shows items in white grey before active
\setbeamertemplate{caption}{\raggedright\insertcaption\par} % don't show "figure 1" in caption
%
% -----------------------------------------------------------------------------
%
\usepackage[utf8x]{inputenc}
\usepackage{amsmath}
\usepackage{amssymb}
%\usepackage{mathrsfs}
\usepackage{mathtools}
\usepackage{graphicx}
\usepackage{tikz}
\usepackage{tcolorbox}
\usepackage{xcolor}
\usepackage{anyfontsize}
%

% -----------------------------------------------------------------------------
%
\pdfpageattr {/Group << /S /Transparency /I true /CS /DeviceRGB>>}
%
% -----------------------------------------------------------------------------
%
%\setcounter{tocdepth}{2}
%\stepcounter{subsection}
%
\patchcmd{\slideentry}{\advance\beamer@xpos by1\relax}{}{}{}
\def\beamer@subsectionentry#1#2#3#4#5{\advance\beamer@xpos by1\relax}%

% -----------------------------------------------------------------------------
%

\title[]{\textbf{Passive monitoring using traffic noise recordings - case study on the Steinachtal Bridge}}
%\subtitle{}
\author[Johannes Salvermoser]{Johannes Salvermoser, C{\'e}line Hadziioannou, Simon C. St{\"a}hler}
%\date{\today}
\institute{Institute for Earth and Environmental Sciences\\ Ludwig Maximilians University Munich}
%
% -----------------------------------------------------------------------------
%
\hypersetup{%
pdftitle    = {EGU 2015},
pdfsubject  = {},
pdfauthor   = {Johannes Salvermoser},
pdfkeywords = {}}
%
% -----------------------------------------------------------------------------
%
\graphicspath{{../Pics/}}
\setbeamertemplate{navigation symbols}{}
%
% =============================================================================
\begin{document}
% =============================================================================
%
\frame{\titlepage}
%
% =============================================================================
%
%\begin{frame}
%\begin{figure}
%\includegraphics[width=0.85\textwidth]{./pics/title.png}
%\end{figure}
%\end{frame}
%
% =============================================================================
%
\section{Overview}
\subsection{Overview}

%\begin{frame}\frametitle{Overview}
%    \begin{center}
%        \begin{tikzpicture}
%            \node[anchor=south west, inner sep=0] at (0,2.5) {\begin{small}
%            		\begin{tcolorbox}[colback=green!5,colframe=salve@blue,title=Contents, width=9cm]
%						\begin{enumerate}
%							\item Measurement setup
%							\item Raw signal observed on the bridge
%							\item Method
%							\item Results
%							\item Reliability tests
%							\item Conclusion and perspective						
%						\end{enumerate}
%					\end{tcolorbox}
%            	\end{small}
%            	};
%            \node[anchor=south west, inner sep=0] at (0,0) {\begin{small}
%            	\begin{tcolorbox}[colback=white,colframe=white,title=, width=9cm]{\textcolor{white}{
%						\item[] Can we do small-scale monitoring on structures, only requiring (ambient) traffic noise?}}
%					\end{tcolorbox}
%            	\end{small}
%            	};
%        \end{tikzpicture}
%    \end{center}
%\end{frame}

\begin{frame}\frametitle{Overview}
    \begin{center}
        \begin{tikzpicture}
            \node[anchor=center] at (0,0) {\begin{small}
            	\begin{tcolorbox}[colback=green!5,colframe=salve@blue,title=Objective, width=9cm]{Is it possible to use ambient and/or traffic noise to monitor small-scale structures?}
					\end{tcolorbox}
            	\end{small}
            	};
        \end{tikzpicture}
    \end{center}
\end{frame}

%\begin{frame}\frametitle{Overview}
%    \begin{center}
%        \begin{tikzpicture}
%            \node[anchor=south west, inner sep=0] at (0,2.5) {\begin{small}
%            		\begin{tcolorbox}[colback=green!5,colframe=salve@blue,title=Contents, width=9cm]
%						\begin{enumerate}
%							\item Measurement setup
%							\item Raw signal observed on the bridge
%							\item Method
%							\item Results
%							\item Reliability tests
%							\item Conclusion and perspective						
%						\end{enumerate}
%					\end{tcolorbox}
%            	\end{small}
%            	};
%            \node[anchor=south west, inner sep=0] at (0,0.3) {\begin{small}
%            	\begin{tcolorbox}[colback=green!5,colframe=salve@blue,title=Objective, width=9cm]
%						\item[] Can we do small-scale monitoring on structures, only requiring (ambient) traffic noise?
%					\end{tcolorbox}
%            	\end{small}
%            	};
%        \end{tikzpicture}
%    \end{center}
%\end{frame}

%
% =============================================================================
%


\section{Setup}


\subsection{Setup}
\begin{frame}\frametitle{Measurement Setup I}
    \begin{center}
        \begin{tikzpicture}
            \node[anchor=south west,inner sep=0] (image) at (0,0.5) {\includegraphics[width=\textwidth]{Figures/setup1.jpg}};
            \node[align=center] at (7.4,1.0) {};
        \end{tikzpicture}
    \end{center}
\end{frame}


\begin{frame}\frametitle{Measurement Setup I}
    \begin{center}
        \begin{tikzpicture}
            \node[anchor=south west,inner sep=0] (image) at (0,0.5) {\includegraphics[width=\textwidth]{Figures/setup2.jpg}};
            \node[align=center] at (7.4,1.0) {};
        \end{tikzpicture}
    \end{center}
\end{frame}


% =============================================================================

\subsection{Setup}
\begin{frame}\frametitle{Measurement Setup II: Steinachtal Bridge}
    \begin{center}
        \begin{tikzpicture}
            \node[anchor=south west,inner sep=0] (image) at (-5,4) {\includegraphics[width=0.55\textwidth]{Figures/bridge_outside2.JPG}};
            \node[anchor=south west, inner sep=0] at (2,2.5) {\includegraphics[width=0.36\textwidth]{Figures/overview.JPG}};
            \node[anchor=south west, inner sep=0] at (3,6) {\includegraphics[width=0.2\textwidth]{Figures/Geophone.JPG}};
        \end{tikzpicture}
    \end{center}
\end{frame}
%
% =============================================================================
%

\section{Raw}
\subsection{Raw}
\begin{frame}{bla}\frametitle{Raw Signal}
	\begin{center}
        \begin{tikzpicture}
            \node[anchor=south west,inner sep=0] (image) at (0,0) 	  
            	 {\includegraphics[width=\linewidth,height=0.89\textheight,keepaspectratio]{Figures/raw1.png}};
            \node[align=center] at (image.center) {};
        \end{tikzpicture}
    \end{center}
\end{frame}

\begin{frame}{bla}\frametitle{Raw Signal}
	\begin{center}
        \begin{tikzpicture}
            \node[anchor=south west,inner sep=0] (image) at (0,0) 	  
            	 {\includegraphics[width=\linewidth,height=0.89\textheight,keepaspectratio]{Figures/raw2.png}};
            \node[align=center] at (image.center) {};
        \end{tikzpicture}
    \end{center}
\end{frame}

\begin{frame}{bla}\frametitle{Raw Signal}
	\begin{center}
        \begin{tikzpicture}
            \node[anchor=south west,inner sep=0] (image) at (0,0) 	  
            	 {\includegraphics[width=\linewidth,height=0.89\textheight,keepaspectratio]{Figures/raw2+arrows.png}};
            \node[align=center] at (image.center) {};
        \end{tikzpicture}
    \end{center}
\end{frame}

\begin{frame}{bla}\frametitle{Raw Signal}
	\begin{center}
        \begin{tikzpicture}
            \node[anchor=south west,inner sep=0] (image) at (0,0) 	  
            	 {\includegraphics[width=\linewidth,height=0.89\textheight,keepaspectratio]{Figures/raw2+arrows.png}};
            \node[align=center] at (image.center) {};
            \node[anchor=center] at (image.center) {
            		\begin{tcolorbox}[colback=green!5,colframe=salve@blue,title=Spectrum, width=7cm]
						\includegraphics[width=\textwidth]{Figures/spectrum1.png}
					\end{tcolorbox}
            };
        \end{tikzpicture}
    \end{center}
\end{frame}

\begin{frame}{bla}\frametitle{Raw Signal}
	\begin{center}
        \begin{tikzpicture}
            \node[anchor=south west,inner sep=0] (image) at (0,0) 	  
            	 {\includegraphics[width=\linewidth,height=0.89\textheight,keepaspectratio]{Figures/raw2+arrows.png}};
            \node[align=center] at (image.center) {};
            \node[anchor=center] at (image.center) {
            		\begin{tcolorbox}[colback=green!5,colframe=salve@blue,title=Spectrum, width=7cm]
						\includegraphics[width=\textwidth]{Figures/spectrum2.png}
					\end{tcolorbox}
            };
        \end{tikzpicture}
    \end{center}
\end{frame}

\begin{frame}{bla}\frametitle{Raw Signal}
	\begin{center}
        \begin{tikzpicture}
            \node[anchor=south west,inner sep=0] (image) at (0,0) 	  
            	 {\includegraphics[width=\linewidth,height=0.89\textheight,keepaspectratio]{Figures/raw2+arrows.png}};
            \node[align=center] at (image.center) {
            \begin{small}
            		\begin{tcolorbox}[colback=green!5,colframe=salve@blue,title=Key Issues, width=7cm]
						\begin{itemize}
							\item Is it better to use the quiet parts of the signal, or those where traffic noise is present?
							\item Can we extract and resolve small velocity variations from the passive signals we observe?
							\item Can we use this information for monitoring structural health?
							\end{itemize}
					\end{tcolorbox}
            	\end{small}
            	};
        \end{tikzpicture}
    \end{center}
\end{frame}


\section{Method}
\subsection{Stretching}
\begin{frame}{bla}\frametitle{CWI \& Cross-correlations}
	\begin{center}
        \begin{tikzpicture}
            \node[anchor=south west,inner sep=0] (image) at (0,0) 	  
            	 {\includegraphics[width=\linewidth,height=0.89\textheight,keepaspectratio]{Figures/hot_cold1.png}};
        \end{tikzpicture}
    \end{center}
\end{frame}

\begin{frame}{bla}\frametitle{CWI \& Cross-correlations}
	\begin{center}
        \begin{tikzpicture}
            \node[anchor=south west,inner sep=0] (image) at (0,0) 	  
            	 {\includegraphics[width=\linewidth,height=0.89\textheight,keepaspectratio]{Figures/hot_cold1.png}};
            \node[anchor=south west,inner sep=0] at (1.3,0.6)
            	 {\includegraphics[width=0.4\linewidth]{Figures/cc.JPG}};  
            \node[anchor=south west,inner sep=0] at (1,2.2) {
            	\begin{small}
            		\begin{tcolorbox}[colback=green!5,colframe=salve@blue,title=, width=5cm]
						\textbf{Hourly} cross-correlations for receiver pairs
					\end{tcolorbox}
            	\end{small}
            };
            \node[anchor=south west,inner sep=0] at (6.8,1) {
            	\begin{small}
            		\begin{tcolorbox}[colback=green!5,colframe=salve@blue,title=, width=4cm]
						\textbf{Unilateral} sources and reverberations \begin{center} $\Downarrow$ \end{center}
						\textbf{No} Green's function retrieval
					\end{tcolorbox}
            	\end{small}
            };        
        \end{tikzpicture}
    \end{center}
\end{frame}

\begin{frame}{bla}\frametitle{CWI \& Cross-correlations}
	\begin{center}
        \begin{tikzpicture}
            \node[anchor=south west,inner sep=0] (image) at (0,0) 	  
            	 {\includegraphics[width=\linewidth,height=0.89\textheight,keepaspectratio]{Figures/hot_cold3.png}};
            \node[align=center] at (image.center) {};
        \end{tikzpicture}
    \end{center}
\end{frame}

\begin{frame}{bla}\frametitle{CWI \& Cross-correlations}
	\begin{center}
        \begin{tikzpicture}
            \node[anchor=south west,inner sep=0] (image) at (0,0) 	  
            	 {\includegraphics[width=\linewidth,height=0.89\textheight,keepaspectratio]{Figures/hot_cold3.png}};
            \node[anchor=center] (statements) at (image.center) {
            	\begin{small}
            		\begin{tcolorbox}[colback=green!5,colframe=salve@blue,title=Statements, width=8cm]
						\begin{itemize}
							\item Traffic noise is required for sufficiently high SNR
							\item Reverberations for natural frequencies around 3 Hz last long enough and can be used for CWI 
							\item Stretching was used to quantify wavespeed changes
							\item Analysis applied to two months of contiunous recordings
						\end{itemize}
					\end{tcolorbox}
            	\end{small}
            };        
        \end{tikzpicture}
    \end{center}
\end{frame}

%\begin{frame}{bla}\frametitle{CWI \& Cross-correlations}
%	\begin{center}
%        \begin{tikzpicture}
%            \node[anchor=south west,inner sep=0] (image) at (0,0) 	  
%            	 {\includegraphics[width=\linewidth,height=0.89\textheight,keepaspectratio]{Figures/hot_cold1.png}};
%            \node[anchor=south west,inner sep=0] (statements) at (0,0.5) {
%            	\begin{small}
%            		\begin{tcolorbox}[colback=green!5,colframe=salve@blue,title=Statements, width=6cm]
%						\begin{itemize}
%							\item Traffic noise is required for sufficiently high SNR
%							\item Reverberations for natural frequencies around 3 Hz last long enough and can be used for CWI 
%						\end{itemize}
%					\end{tcolorbox}
%            	\end{small}
%            };
%            \node[anchor=south west,inner sep=0] at (6.5,1.5) {
%            		\begin{tcolorbox}[colback=green!5,colframe=salve@blue,title=Spectrum, width=4.5cm]
%						\includegraphics[width=\textwidth]{Figures/spectrum.png}
%					\end{tcolorbox}
%            };
%        \end{tikzpicture}
%    \end{center}
%\begin{figure}
%\end{figure}
%\end{frame}

% =============================================================================
\section{Results}
\subsection{Results}
\begin{frame}{bla}\frametitle{Observed Results}
	\begin{center}
        \begin{tikzpicture}[overlay]
            \node[anchor=south west,inner sep=0] (image) at (-6,-3.4) {\includegraphics[width=\linewidth,height=0.89\textheight,keepaspectratio]{Figures/whole1_flip.png}};
			 \node[anchor=south west,inner sep=0] at (-4.5,2.5) {\textbf{\color{blue} Velocity variation $\frac{\Delta v}{v}$}};
        \end{tikzpicture}
    \end{center}
\end{frame}

\begin{frame}{bla}\frametitle{Observed Results}
	\begin{center}
        \begin{tikzpicture}[overlay]
            \node[anchor=south west,inner sep=0] (image) at (-6,-3.4) {\includegraphics[width=\linewidth,height=0.89\textheight,keepaspectratio]{Figures/whole2_flip.png}};
			 \node[anchor=south west,inner sep=0] at (-4.5,2.5) {\textbf{\color{blue} Velocity variation $\frac{\Delta v}{v}$}};
			 \node[anchor=south west,inner sep=0] at (1.1,2.5) {\textbf{\color{Green} Temperature}};
			 \node[anchor=south west,inner sep=0] (image) at (-0.3,2.5) {\includegraphics[width=0.06\linewidth,keepaspectratio]{Figures/vs.png}};
        \end{tikzpicture}
    \end{center}
\end{frame}

\begin{frame}{bla}\frametitle{Observed Results}
	\begin{center}
        \begin{tikzpicture}[overlay]
            \node[anchor=south west,inner sep=0] (image) at (-6,-3.4) {\includegraphics[width=\linewidth,height=0.89\textheight,keepaspectratio]{Figures/whole2.png}};
			 \node[anchor=south west,inner sep=0] at (-4.5,2.5) {\textbf{\color{blue} Velocity variation \color{red} $\frac{\Delta t}{t}$}};
			 \node[anchor=south west,inner sep=0] at (1.1,2.5) {\textbf{\color{Green} Temperature}};
			 \node[anchor=south west,inner sep=0] (image) at (-0.3,2.5) {\includegraphics[width=0.06\linewidth,keepaspectratio]{Figures/vs.png}};
        \end{tikzpicture}
    \end{center}
\end{frame}

\begin{frame}{bla}\frametitle{Observed Results}
	\begin{center}
        \begin{tikzpicture}[overlay]
            \node[anchor=south west,inner sep=0] (image) at (-6,-3.4) {\includegraphics[width=\linewidth,height=0.89\textheight,keepaspectratio]{Figures/whole3_altern.png}};
			 \node[anchor=south west,inner sep=0] at (-4.5,2.5) {\textbf{\color{blue} Velocity variation $\frac{\Delta t}{t}$}};
			 \node[anchor=south west,inner sep=0] at (1.1,2.5) {\textbf{\color{Green} Temperature}};
			 \node[anchor=south west,inner sep=0] (image) at (-0.3,2.5) {\includegraphics[width=0.06\linewidth,keepaspectratio]{Figures/vs.png}};
        \end{tikzpicture}
    \end{center}
\end{frame}

% =============================================================================

\subsection{Velocity variations II}

\begin{frame}{bla}\frametitle{March 2012}
	\begin{center}
        \begin{tikzpicture}[overlay]
            \node[anchor=south west,inner sep=0] (image) at (-6.1,-3.5) 	{\includegraphics[width=\linewidth,height=0.89\textheight,keepaspectratio]{Figures/stretch_MAR1.png}};
            \node[anchor=south west,inner sep=0] at (-6,2.5) {\textbf{\color{blue} Velocity variation $\frac{\Delta t}{t}$}};
        \end{tikzpicture}
    \end{center}
\end{frame}

\begin{frame}{bla}\frametitle{March 2012}
\transfade
	\begin{center}
        \begin{tikzpicture}[overlay]
            \node[anchor=south west,inner sep=0] (image) at (-6.1,-3.5) 	{\includegraphics[width=\linewidth,height=0.89\textheight,keepaspectratio]{Figures/stretch_MAR2.png}};
            \node[anchor=south west,inner sep=0] at (-6,2.5) {\textbf{\color{blue} Velocity variation $\frac{\Delta t}{t}$}};
            \node[anchor=south west,inner sep=0] at (-5.2,2) {\textbf{\color{black!40} Deviation}};
            \node[anchor=south west,inner sep=0] at (-6,1.5) {\textbf{\color{black!40} (32 receiver pairs)}};
        \end{tikzpicture}
    \end{center}
\end{frame}

\begin{frame}{bla}\frametitle{March 2012}
\transfade
	\begin{center}
        \begin{tikzpicture}[overlay]
            \node[anchor=south west,inner sep=0] (image) at (-6.1,-3.5) 	{\includegraphics[width=\linewidth,height=0.89\textheight,keepaspectratio]{Figures/stretch_MAR3.png}};
            \node[anchor=south west,inner sep=0] at (-6,2.5) {\textbf{\color{blue} Velocity variation $\frac{\Delta t}{t}$}};
            \node[anchor=south west,inner sep=0] at (-5.2,2) {\textbf{\color{black!40} Deviation}};
            \node[anchor=south west,inner sep=0] at (-6,1.5) {\textbf{\color{black!40} (32 receiver pairs)}};
            \node[anchor=south west,inner sep=0] at (-1,2.5) {\textbf{\color{Green} Temperature}};
        \end{tikzpicture}
    \end{center}
\end{frame}

\begin{frame}{bla}\frametitle{March 2012}
\transfade
	\begin{center}
        \begin{tikzpicture}[overlay]
            \node[anchor=south west,inner sep=0] (image) at (-6.1,-3.5) 	{\includegraphics[width=\linewidth,height=0.89\textheight,keepaspectratio]{Figures/stretch_MAR4.png}};
            \node[anchor=south west,inner sep=0] at (-6,2.5) {\textbf{\color{blue} Velocity variation $\frac{\Delta t}{t}$}};
            \node[anchor=south west,inner sep=0] at (-5.2,2) {\textbf{\color{black!40} Deviation}};
            \node[anchor=south west,inner sep=0] at (-6,1.5) {\textbf{\color{black!40} (32 receiver pairs)}};
            \node[anchor=south west,inner sep=0] at (-1,2.5) {\textbf{\color{Green} Temperature}};
        \end{tikzpicture}
    \end{center}
\end{frame}

\begin{frame}{bla}\frametitle{March 2012}
\transfade
	\begin{center}
        \begin{tikzpicture}[overlay]
            \node[anchor=south west,inner sep=0] (image) at (-6.1,-3.5) 	{\includegraphics[width=\linewidth,height=0.89\textheight,keepaspectratio]{Figures/stretch_MAR5.png}};
            \node[anchor=south west,inner sep=0] at (-6,2.5) {\textbf{\color{blue} Velocity variation $\frac{\Delta t}{t}$}};
            \node[anchor=south west,inner sep=0] at (-5.2,2) {\textbf{\color{black!40} Deviation}};
            \node[anchor=south west,inner sep=0] at (-6,1.5) {\textbf{\color{black!40} (32 receiver pairs)}};
            \node[anchor=south west,inner sep=0] at (-1,2.5) {\textbf{\color{Green} Temperature}};
            \node[anchor=south west,inner sep=0] at (3,2.5) {\textbf{\color{red!40} Simulated core}};
            \node[anchor=south west,inner sep=0] at (3.2,2) {\textbf{\color{red!40} temperature}};
        \end{tikzpicture}
    \end{center}
\end{frame}

\begin{frame}{bla}\frametitle{March 2012}
\transfade
	\begin{center}
        \begin{tikzpicture}[overlay]
            \node[anchor=south west,inner sep=0] (image) at (-6.1,-3.5) 	{\includegraphics[width=\linewidth,height=0.89\textheight,keepaspectratio]{Figures/stretch_MAR5.png}};
            \node[anchor=south west,inner sep=0] at (-6,2.5) {\textbf{\color{blue} Velocity variation $\frac{\Delta t}{t}$}};
            \node[anchor=south west,inner sep=0] at (-5.2,2) {\textbf{\color{black!40} Deviation}};
            \node[anchor=south west,inner sep=0] at (-6,1.5) {\textbf{\color{black!40} (32 receiver pairs)}};
            \node[anchor=south west,inner sep=0] at (-1,2.5) {\textbf{\color{Green} Temperature}};
            \node[anchor=south west,inner sep=0] at (3,2.5) {\textbf{\color{red!40} Simulated core}};
            \node[anchor=south west,inner sep=0] at (3.2,2) {\textbf{\color{red!40} temperature}};
            \node[anchor=south west,inner sep=0] at (-3.5,-2.3) {\begin{tcolorbox}[colback=green!5,colframe=salve@blue,title=Overall Results,width=7cm]
            		    \begin{small}
							\begin{tabular}{ll}
								$\frac{\Delta v}{v}$  & -1.5$\%$ to +2.1$\%$\\
								&\\
								temperatures & +14${\textdegree}$C to -23${\textdegree}$C\\
								&\\
								average rate & 0.064 $\frac{\%}{{\textdegree}C}$\\
								&\\
								diffusion lag & $\approx$ 3 hours
							
							\end{tabular}
            			\end{small}
					\end{tcolorbox}};
		\end{tikzpicture}
    \end{center}
\end{frame}


\section{Tests}
\subsection{thermalex}
\begin{frame}
	\frametitle{Reliability Tests I}
	\begin{center}
		\begin{tikzpicture}
            \node[anchor=south west,inner sep=0] (box1) at (0,-0.13) {
            		\begin{tcolorbox}[colback=green!5,colframe=salve@blue,title=Thermal expansion,width=8cm]
            		    \begin{small}
							\begin{itemize}
								\item Varying inter-receiver distances cause traveltime variations
								\item $\frac{\Delta t}{t} = - \frac{\Delta v}{v}$
								\item[] $\Rightarrow$ Effect in the order of \textbf{6-14 $\mathbf{\cdot 10^{-4}}$ $\frac{\%}{\textdegree{C}}$} for steel-reinforced concrete (cf. \textit{Keller, T. and C. Menn} (1992))
							\end{itemize}
            			\end{small}
					\end{tcolorbox}};
            \node[align=center] at (10,2) {\textcolor{white}{\fontsize{40}{50} $\checkmark$}};
            \node[align=center] at (10,3) {\color{white} reference range:\\ 
            		\color{white} -1.5 to +2.3 \%};             
        \end{tikzpicture}
   \end{center}
\end{frame}

\begin{frame}
	\frametitle{Reliability Tests I}
	\begin{center}
		\begin{tikzpicture}
            \node[anchor=south west,inner sep=0] (box1) at (0,0) {
            		\begin{tcolorbox}[colback=green!5,colframe=salve@blue,title=Thermal expansion,width=8cm]
            		    \begin{small}
							\begin{itemize}
								\item Varying inter-receiver distances cause traveltime variations
								\item $\frac{\Delta t}{t} = - \frac{\Delta v}{v}$
								\item[] $\Rightarrow$ Effect in the order of \textbf{6-14 $\mathbf{\cdot 10^{-4}}$ $\frac{\%}{\textdegree{C}}$} for steel-reinforced concrete (cf. \textit{Keller, T. and C. Menn} (1992))	
							\end{itemize}
            			\end{small}
					\end{tcolorbox}};
            \node[align=center] at (10,2) {\textcolor{green}{\fontsize{40}{50} $\checkmark$}};
            \node[align=center] at (10,3) {
            	 reference range:\\ 
            		-1.5 to +2.3 \%            
            	 };            
        \end{tikzpicture}
   \end{center}
\end{frame}

%\subsection{mathtest}
\begin{frame}
	\frametitle{Reliability Tests II}
	\begin{center}
		\begin{tikzpicture}       
            \node[anchor=south west,inner sep=0] (box2) at (0,0) {
            		\begin{tcolorbox}[colback=green!5,colframe=salve@blue,title=Mathematical test of instrument 		     
            		                  stability,width=8cm]
            		    \begin{small}
							\begin{itemize}
								\item Calculate temperature dependence of geophones
								\item Deconvolve impulse response and convolve with slightly different response ($\Delta f_c$ = 											0.2 Hz)
								\item[] $\Rightarrow$ Correlation of original and simulated signal yielded an apparent delay of 												\textbf{4.8 $\mathbf{\cdot 10^{-2}}$ \%}.
							\end{itemize}
            			\end{small}
					\end{tcolorbox}};
            \node[align=center] at (10,2) {\textcolor{white}{\fontsize{40}{50} $\checkmark$}};
            \node[align=center] at (10,3) {\color{white} reference range:\\ 
            		\color{white} -1.5 to +2.3 \%};              
        \end{tikzpicture}
    \end{center}
\end{frame}

\begin{frame}
	\frametitle{Reliability Tests II}
	\begin{center}
		\begin{tikzpicture}       
            \node[anchor=south west,inner sep=0] (box2) at (0,0) {
            		\begin{tcolorbox}[colback=green!5,colframe=salve@blue,title=Mathematical test of instrument 		     
            		                  stability,width=8cm]
            		    \begin{small}
							\begin{itemize}
								\item Calculate temperature dependence of geophones
								\item Deconvolve impulse response and convolve with slightly different response ($\Delta f_c$ = 											0.2 Hz)
								\item[] $\Rightarrow$ Correlation of original and simulated signal yielded an apparent delay of 												\textbf{4.8 $\mathbf{\cdot 10^{-2}}$ \%}.
							\end{itemize}
            			\end{small}
					\end{tcolorbox}};
            \node[align=center] at (10,2) {\textcolor{green}{\fontsize{40}{50} $\checkmark$}};
            \node[align=center] at (10,3) {
            	 reference range:\\ 
            		-1.5 to +2.3 \%            
            	 };  
        \end{tikzpicture}
    \end{center}
\end{frame}
	
\section{Conclusions}
\subsection{Conclusions}
\begin{frame}
	\frametitle{Conclusions}
	\begin{center}
		\begin{tikzpicture}       
            \node[anchor=south west,inner sep=0] (box2) at (0,0) {
            		\begin{tcolorbox}[colback=green!5,colframe=salve@blue,title=,width=10cm]
            		    \begin{small}
							\begin{itemize}
								\item Resolution of velocity variations is possible via cross-correlations from ambient traffic noise on a bridge
								\item Captured small velocity variations caused by temperature fluctuations:\\
									 relative velocity $\frac{\Delta v}{v}$: -1.5$\%$ to +2.1$\%$\\
									 temperature range: +14${\textdegree}$C to -23${\textdegree}$C
								\item Strong correlation between temperature and $\frac{\Delta v}{v}$ series
								\item Advantages: high temporal resolution, high accuracy, low logistical effort
							\end{itemize}
            			\end{small}
					\end{tcolorbox}};
         \end{tikzpicture}
    \end{center}
\end{frame}

\subsection{Outlook}
\begin{frame}
	\frametitle{Perspective}
	\begin{center}
		\begin{tikzpicture}       
            \node[anchor=south west,inner sep=0] (box2) at (0,0) {
            		\begin{tcolorbox}[colback=green!5,colframe=salve@blue,title=Aspired Project,width=10cm]
            		    \begin{small}
							\begin{itemize}
								\item Long-term ($>$ 1 year) monitoring of a highway bridge
								\item[] $\Rightarrow$ improve characterization of temperature effect  
								\item Damage-scenario tests on sample bodies and expired structures
								\item Numerical simulations
								\item[] $\Rightarrow$ confirm reliability of damage detection
							\end{itemize}
            			\end{small}
					\end{tcolorbox}};
         \end{tikzpicture}
    \end{center}
\end{frame}
%-------------------------------------------------------------------------------------------
%-------------------------------------------------------------------------------------------
%Questions slide

\begin{frame}
	\frametitle{Questions}
	\begin{center}
		\begin{tikzpicture}       
            \node[anchor=south west,inner sep=0] (image) at (-3,-2) {\includegraphics[width=0.3\textwidth]{Figures/questions.jpeg}};
            \node[anchor=south west,inner sep=0] (thanks) at (1,1) {\textcolor{salve@blue}{\Huge{Thank you!}}};
         \end{tikzpicture}
    \end{center}
\end{frame}


%
% ============================================================================
\section{Extra}
\subsection{extra slides}
\begin{frame}
	\frametitle{Daily Sunlight}
	\begin{center}
		\includegraphics[width=\linewidth]{Figures/whole3.png}
    \end{center}
\end{frame}

\begin{frame}
	\frametitle{Stretching Method}
	\begin{center}
		\includegraphics[width=\linewidth]{Figures/stretchingMethod.png}
    \end{center}
\end{frame}

\begin{frame}
	\frametitle{Eigenfrequency Evolution}
	\begin{center}
		\includegraphics[width=\linewidth]{Figures/eigenfreq_evolution.png}
    \end{center}
\end{frame}

\begin{frame}
	\frametitle{Instrument Stability Test}
	\begin{center}
		\includegraphics[width=.85\linewidth]{Figures/freq_resp2.JPEG}
    \end{center}
\end{frame}
%
% =============================================================================
\end{document}
% =============================================================================
% EOF
% =============================================================================
